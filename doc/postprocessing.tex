\chapter{Post-Processing Tools}\label{sec:postprocessing}

%%%%%%%%%%%%%%%%%%%%%%%%%%%%%%%%%%%%%%%%%%%%%%%%%%%%%%%%%%%%%%%%%%%%%%%%%%%%%%%%
%%%%%%%%%%%%%%%%%%%%%%%%%%%%%%%%%%%%%%%%%%%%%%%%%%%%%%%%%%%%%%%%%%%%%%%%%%%%%%%%
\section{Averages}

See file {\tt dns/tools/statistics/averages}.

Allows for conditional analysis.

%%%%%%%%%%%%%%%%%%%%%%%%%%%%%%%%%%%%%%%%%%%%%%%%%%%%%%%%%%%%%%%%%%%%%%%%%%%%%%%%
%%%%%%%%%%%%%%%%%%%%%%%%%%%%%%%%%%%%%%%%%%%%%%%%%%%%%%%%%%%%%%%%%%%%%%%%%%%%%%%%
\section{Probability density functions}

See file {\tt dns/tools/statistics/pdfs}.

Allows for conditional analysis.

%%%%%%%%%%%%%%%%%%%%%%%%%%%%%%%%%%%%%%%%%%%%%%%%%%%%%%%%%%%%%%%%%%%%%%%%%%%%%%%%
%%%%%%%%%%%%%%%%%%%%%%%%%%%%%%%%%%%%%%%%%%%%%%%%%%%%%%%%%%%%%%%%%%%%%%%%%%%%%%%%
\section{Conditional analysis}

To be developed.

%%%%%%%%%%%%%%%%%%%%%%%%%%%%%%%%%%%%%%%%%%%%%%%%%%%%%%%%%%%%%%%%%%%%%%%%%%%%%%%%
%%%%%%%%%%%%%%%%%%%%%%%%%%%%%%%%%%%%%%%%%%%%%%%%%%%%%%%%%%%%%%%%%%%%%%%%%%%%%%%%
\section{Spatially coarse grained data at full time resolution}

Data can be saved at full temporal resolution with a user-defined spatial coarse graining. In that mode, on top of the standard output, a single file is generated for each horizontal  data point per restart. This allows to maintain full flexibility and at the same time to  avoid parallel I/O at this stage.
%
For small strides (see table), this approach may produce a large number of files which can cause problems if the data is not converted to netCDF immediately after a simulation is run.
%
The files can be merged into a netCDF file with the script \texttt{scripts/python/tower2nc.py}.

{
\centering
\setlength{\tabcolsep}{0pt}
\footnotesize

\rowcolors{1}{gray!25}{gray!10}
%
\begin{longtable}{p{0.15\textwidth} p{0.3\textwidth} p{0.55\textwidth}}
%
\multicolumn{3}{>{\columncolor{lightblue}}c}{\normalsize\bf [SaveTowers]}\\
%
\tt Stride & {\it value1, value2, value3} & Strides along the directions $Ox$, $Oy$ and $Oz$.
\end{longtable}

}

For example, {\tt Stride=0,0,0} would not save any data; {\tt Stride=1,1,1} would save all points, but it more efficient to use {\tt Iteration.Restart=1}; {\tt Stride=16, 1,16} would save vertical profiles every 16x16 horizontal point. A Stride of 0 has no effect whatsoever (no output will be generated). \textit{Note: vertical corse graining is not tested, but should work. }

\begin{description}
\item[File name] ~\\
  \textbf{The file name contains crucial information} which is not stored elsewhere.
  Hence, files should not be renamed.
  Output files are named by the following scheme \\[0.25em]
  \begin{tabular}{r c r c r c r c r c r}
    tower&.&iloc&$\times$&kloc  &.&$t_\mathrm{start}$&-&$t_\mathrm{end}$&.&ivar\\
    tower&.&000015&$\times$&000015&.&000001&-&000010&.&1
  \end{tabular}\\[0.25em]
  %
  Only one scalar is supported, and the variables (ivar) are \\[0.25em]
  \centerline{
    \begin{tabular}{r l}
      1--3 & velocities ($u,v,w$) \\
      4    & pressure ($p$) \\
      5    & scalar1 ($s_1$)\\
    \end{tabular}
  }
  In case of multiple scalars, only the first one will be output.
%
\item[File structure] ~\\
  The file consists of $nt=it_\mathrm{end}-it_\mathrm{start}+1$ records. Irrespective of the
  iteration being an integer, all data is saved in double precision real format,
  i.e. one record has $(ny+2)*8\,Bytes$.
  The whole file has $8* (it_\mathrm{end}-it_\mathrm{start}+1)  * (ny+2)\,Bytes$. \\
  \begin{figure}
    \begin{centering}
      Inner direction of write $\rightarrow$ \\[0.25em]
      \begin{tabular}{| l | l | c | c | c | c | }
        \hline
        $it_\mathrm{start}$ & $t(it_\mathrm{start})$ & $v_{ivar}(y_1)$ & $v_{ivar}(y_2)$ & \ldots & $v_{ivar}(ny)$ \\     \hline
        $it_\mathrm{start}+1$ & $t(it_\mathrm{start}+1)$ & $v_{ivar}(y_1)$ & $v_{ivar}(y_2)$ & \ldots & $v_{ivar}(ny)$ \\
        $\qquad$\vdots & $\qquad$\vdots & \vdots & \vdots & ~ & \vdots \\
        $it_\mathrm{end}-1$   & $t(it_\mathrm{end}-1)$ & $v_{ivar}(y_1)$ & $v_{ivar}(y_2)$ & \ldots & $v_{ivar}(ny)$ \\ \hline
        $it_\mathrm{end}$   & $t(it_\mathrm{end})$ & $v_{ivar}(y_1)$ & $v_{ivar}(y_2)$ & \ldots & $v_{ivar}(ny)$ \\
        \hline
      \end{tabular}\\
    \end{centering}
  \caption{Internal organization of tower files.}
   \end{figure}
  \item[netCDF output]~ \\
    The python script \[\texttt{tlab/scripts/python/tower2nc.py}\] is available to bundle tower files from one restart into a single netCDF file which
    is then self-descriptive through its Meta-data. The script handles tower files of multiple restarts, but it will generate one netCDF file per restart.
    The python script is executed in the directory where the tower file resides.
    It relies on
    \begin{itemize}
      \item availability of \textbf{all} tower files belonging to one restart,
      \item the \texttt{grid} file used for the simulation,
      \item the \texttt{dns.ini} used for the simulation (restart etc. does not matter, but the value of \texttt{Stride}
        in the section \texttt{[SaveTowers]} must not change),
      \item a properly installed \texttt{netCDF4-python} library. (On ZMAW computers, it may be necessary to load a particular
        python module).

    \end{itemize}
    \textbf{The script is not thread-safe}, i.e. it may only be run once at the same time
    one the same system. This is ascertained by a lock in the form of an empty file which is touched
    in the working directory.
    %
    \par
    %
    The script moves tower files to a directory named \texttt{towerdump\_}$<$\texttt{TIME\_STAMP}$>$. Once the script
    exited successfully and integrity of the netCDF file was checked, this directory may be tarred and archived or removed.
    %
    Should, for any reason, the script exit before successful completion, the files that were already moved to the \texttt{towerdump}
    \textbf{must be copied back} before the script is run again.
    %
\end{description}

%%%%%%%%%%%%%%%%%%%%%%%%%%%%%%%%%%%%%%%%%%%%%%%%%%%%%%%%%%%%%%%%%%%%%%%%%%%%%%%%
%%%%%%%%%%%%%%%%%%%%%%%%%%%%%%%%%%%%%%%%%%%%%%%%%%%%%%%%%%%%%%%%%%%%%%%%%%%%%%%%
\section{Two-point statistics}
\sloppy

See file {\tt dns/tools/statistics/spectra}. Based on package FFTW \cite{Frigo:2005}.

Given two scalar fields $\{a_{nm}:\,n=1,\ldots,N,\,m=1,\ldots,M\}$ and similarly
$b_{nm}$, we calculate the one-dimensional co-spectra
$\{E^x_0,\,E^x_1,\,\ldots,\,E^x_{N/2}\}$ and
$\{E^z_0,\,E^z_1,\,\ldots,\,E^z_{M/2}\}$ normalized such that
\begin{equation}
\langle ab\rangle = E^x_0+2\sum_1^{N/2-1}E^x_n+E^x_{N/2} = E^z_1+2\sum_0^{M/2-1}E^z_m+E^z_{M/2}
\end{equation}
The mean value is removed, such that the left-hand side is $\langle
a'b'\rangle$. The Nyquist frequency energy content $E^x_{N/2}$ and $E^z_{M/2}$
is not written to disk, only the $N/2$ values
$\{E^x_0,\,E^x_1,\,\ldots,\,E^x_{N/2-1}\}$ and the $M/2$ values
$\{E^z_0,\,E^z_1,\,\ldots,\,E^z_{M/2-1}\}$. When $a\equiv b$, then we obtain the
power spectral density.

The sum above can be interpreted as the trapezoidal-rule approximation to the
integral $(L/2\pi)\int_0^{\kappa_c}2E(\kappa)\mathrm{d}\kappa$, where
$\kappa_c=\pi/h$ is the Nyquist frequency, $\Delta \kappa=\kappa_c/(N/2)=2\pi/L$
is the uniform wavenumber spacing, $h=L/N$ is the uniform grid spacing and $L$
is the domain size. Hence, the physical spectral function at wavenumber $\kappa_n=
n\Delta \kappa$ (equivalently, wavelength $L/n$) is $2E_n/\Delta \kappa$.

Due to the relatively large size of the files, we split the calculations is the
auto-spectra and the cross-spectra. The corresponding files containing the
one-dimensional spectra along the direction $Ox$ are {\tt xsp} and {\tt xCsp},
respectively, and similarly along the direction $Oz$. The two-dimensional
co-spectra $E_{nm}$ can also be written to disk, though the additional memory
requirement can be a difficulty.

The one-dimensional cross-correlations $\{C^x_0,\,C^x_1,\,\ldots,\,C^x_{N-1}\}$
and $\{C^z_0,\,C^z_1,\,\ldots,\,C^z_{M-1}\}$ are normalized by
$a_\mathrm{rms}b_\mathrm{rms}$, so that $C^x_0 = C^z_0 =1$ when $a\equiv b$ and
we calculate the auto-correlations. The auto-correlations are even functions and
therefore only $\{C^x_0,\,C^x_1,\,\ldots,\,C^x_{N/2-1}\}$ and
$\{C^z_0,\,C^z_1,\,\ldots,\,C^z_{M/2-1}\}$ are written to disk (note that we
also dropped the last term $C^x_{N/2}$ and $C^z_{M/2}$.)

The corresponding files containing the one-dimensional cross-correlations along
the direction $Ox$ are {\tt xcr} and {\tt xCcr}, and similarly along the
direction $Oz$. The two-dimensional cross-correlation $C_{nm}$ can also be
written to disk, though the additional memory requirement can be a difficulty.

Both form a Fourier pair according to
\begin{equation*}
E_k=\frac{1}{N}\sum_0^{N-1}(a_\mathrm{rms}b_\mathrm{rms}C_n)\exp(-i\omega_kn)\;,\qquad
a_\mathrm{rms}b_\mathrm{rms}C_n=\sum_0^{N-1}E_k\exp(i\omega_kn)\;,
\end{equation*}
where $\{\omega_k=(2\pi/N)k:\, k = 0,\ldots,N-1\}$ is the scaled wavenumber and
$i=\sqrt{-1}$ is the imaginary unit. Therefore,
\begin{equation}
\frac{1}{N}\sum_0^{N-1}C_n=\frac{E_0}{a_\mathrm{rms}b_\mathrm{rms}} \;,
\end{equation}
relation that can be used to relate integral scales $\ell$ to the Fourier mode
$E_0$, as follows. First, for the auto-correlation function, we can re-write
\begin{equation}
\frac{1}{N/2}\sum_0^{N/2-1}C_n=\frac{E_0}{a_\mathrm{rms}^2}+\frac{1-C_{N/2}}{N}
\end{equation}
because
\begin{equation*}
\frac{1}{N}\sum_0^{N-1}C_n = \frac{1}{N}\left(\sum_0^{N/2-1}C_n
+C_{N/2}+\sum_{N/2+1}^{N-1}C_n\right) = \frac{1}{N}\left(2\sum_0^{N/2-1}C_n
+C_{N/2}-1\right) \;,
\end{equation*}
since, from periodicity, $C_N=C_0=1$ and, from the symmetry of the
auto-correlation sequence, $\sum_{N/2+1}^{N}C_n=\sum_0^{N/2-1}C_n$. Therefore,
if we use a trapezoidal rule to define the integral length scale as
\begin{equation}
\ell=h\left(\frac{C_0+C_{N/2-1}}{2}+\sum_1^{N/2-2}C_n\right)\;,
\end{equation}
where $h=L/N$ is the grid spacing and $L$ is the domain size, we obtain
\begin{equation}
\ell=\frac{L}{2}\left(\frac{E_0}{a_\mathrm{rms}^2}-\frac{2C_{N/2}}{N}\right)
\simeq \frac{L}{2a_\mathrm{rms}^2}E_0\;.
\end{equation}
This result applies to both directions $Ox$ and $Oz$, providing relations
between $\ell^x$ and $E^x$, and $\ell^z$ and $E^z$. Each case needs to use the
corresponding domain size, $L^x$ and $L^z$.

These relations show that the integral length scales can be obtained directly
from the spectral information without the need to calculate the correlation
functions. However, the statistical convergence of those integral scales might
be too poor and alternative definitions might be more useful. Also, correlation
functions provide information about the degree of de-correlation achieved with a
particular domain size, and about the structural organization of the flow in
terms of different properties.

%%%%%%%%%%%%%%%%%%%%%%%%%%%%%%%%%%%%%%%%%%%%%%%%%%%%%%%%%%%%%%%%%%%%%%%%%%%%%%%%
%%%%%%%%%%%%%%%%%%%%%%%%%%%%%%%%%%%%%%%%%%%%%%%%%%%%%%%%%%%%%%%%%%%%%%%%%%%%%%%%
\section{Summary of budget equations for second-order moments}

\newcommand{\pt}[2] {\partial_{#2}{#1}}
\newcommand{\ptp}[2]{\partial_{#2}\left(#1\right)}
% \newcommand{\ptp}[1]{\partial_t\left(#1\right)}
% \newcommand{\pjp}[1]{\partial_j\left(#1\right)}
\newcommand{\dvp}[1]{\nabla\negmedspace\cdot\negthinspace\left(#1\right)}
\newcommand{\dv} [1]{\nabla\negmedspace\cdot\negthinspace#1}
\newcommand{\avr}[1]{\overline{#1}}
\newcommand{\avf}[1]{\widetilde{#1}}

Let $f$ and $g$ be fluid properties per unit mass. We use the Reynolds decomposition
\begin{equation}
  f = \avr{f}+f' \;,
\end{equation}
and the Favre decomposition
\begin{equation}
  f = \avf{f}+f'' \;,
\end{equation}
where
\begin{equation}
  \avf{f} = \avr{\rho f}/\avr{\rho} \;.
\end{equation}
Favre decomposition proves convenient in variable-density flows. The overbar indicates ensemble average, which is approximated by spatial average, or temporal average, or both, depending on the  configuration. We will refer to it as Reynolds average, and use the term Favre average to refer to the density-weighted averages (property per unit volume). Note that $\avf{f''}=0$ and $\avr{f'}=0$, but
\begin{equation}
  \avr{f''}=\avr{f}-\avf{f} \;,
\end{equation}
which only zero in constant-density flows, when $\avr{f}$ and $\avf{f}$ coincide.
The covariances satisfy
\begin{align}
  \avr{f'g'}   &= \avr{fg'}  =\avr{f'g}  =\avr{f'g''} =\avr{f''g'} =\avr{fg}-\avr{f}\avr{g} \;,\\
  \avf{f''g''} &= \avf{fg''} =\avf{f''g} =\avf{f'g''} =\avf{f''g'} =\avf{fg}-\avf{f}\avf{g} \;,\\
  \avr{fg}-\avr{f}\avf{g} &=\avr{f''g'}-\avr{f}\,\avr{g''}\;.
\end{align}

We consider evolution equations of the following form:
\begin{subequations}
  \begin{align}
    \ptp{\rho f}{t} + \dvp{\rho f\mathbf{v}}& =-\dv{\mathbf{F}} + \rho S_f \;,\\
    \ptp{\rho g}{t} + \dvp{\rho g\mathbf{v}}& =-\dv{\mathbf{G}} + \rho S_g \;.
  \end{align}
\end{subequations}
Multiplying the first equation by $g$ and the second by $f$ and adding them, one finds the evolution equation for the product $\rho f g$:
\begin{equation}\label{equ:fg}
  \begin{split}
    \ptp{\rho fg}{t} + \dvp{\rho fg\mathbf{v}}=&-g\dv{\mathbf{F}}-f\dv{\mathbf{G}}\\
    &+ \rho gS_f + \rho fS_g\;,
  \end{split}
\end{equation}
having used the equation of conservation of mass
\begin{equation}
  \pt{\rho}{t} + \dvp{\rho\mathbf{v}}=0
\end{equation}
in the left hand side.

The evolution equation for the mean properties can be written as
\begin{subequations}
  \begin{align}
    \ptp{\avr{\rho}\avf{f}}{t} + \dvp{\avr{\rho}\avf{f}\avf{\mathbf{v}}} &=
    -\dvp{\avr{\mathbf{F}}+\avr{\rho}\avf{f''\mathbf{v}''}} + \avr{\rho}\avf{S_f} \;,\\
    \ptp{\avr{\rho}\avf{g}}{t} + \dvp{\avr{\rho}\avf{g}\avf{\mathbf{v}}} &=
    -\dvp{\avr{\mathbf{G}}+\avr{\rho}\avf{g''\mathbf{v}''}} + \avr{\rho}\avf{S_g} \;.
  \end{align}
\end{subequations}
Multiplying the first equation by $\avf{g}$ and the second by $\avf{f}$ and adding them, one finds the evolution equation for the product $\avr{\rho}\avf{f}\avf{g}$:
\begin{equation}\label{equ:avfg}
  \begin{split}
    \ptp{\avr{\rho} \avf{f}\avf{g}}{t} + \dvp{\avr{\rho} \avf{f}\avf{g}\avf{\mathbf{v}}}=
    &-\avf{g}\dv{\avr{\mathbf{F}}}-\avf{f}\dv{\avr{\mathbf{G}}}\\
    &-\dvp{\avr{\rho}\avf{f''\mathbf{v}''}\avf{g}+\avr{\rho}\avf{g''\mathbf{v}''}\avf{f}}\\
    &+ \avr{\rho}\avf{\mathbf{v}''f''}\cdot\nabla \avf{g}
     + \avr{\rho}\avf{\mathbf{v}''g''}\cdot\nabla \avf{f}\\
    &+ \avf{g}\avr{S_f} + \avf{f}\avr{S_g}\;,
  \end{split}
\end{equation}
having used the equation of conservation of mass
\begin{equation}
  \pt{\avr{\rho}}{t} + \dvp{\avr{\rho}\avf{\mathbf{v}}}=0
\end{equation}
in the left hand side.

Averaging Eq.~(\ref{equ:fg}) and substracting Eq.~(\ref{equ:avfg}) from it, one finds
\begin{equation}\label{equ:covfg}
  \begin{split}
    \ptp{\avr{\rho} \avf{f''g''}}{t} + \dvp{\avr{\rho} \avf{f''g''}\avf{\mathbf{v}}} =
    &-\dvp{\avr{\rho}\avf{\mathbf{v}''f''g''} + \avr{\mathbf{F}'g''} + \avr{\mathbf{G}'f''}} \\
    &+\avr{\mathbf{F}'\cdot\nabla g''} + \avr{\mathbf{G}'\cdot\nabla f''} \\
    &-\avr{\rho}\avf{\mathbf{v}''f''}\cdot\nabla\avf{g} -\avr{\rho}\avf{\mathbf{v}''g''}\cdot\nabla\avf{f} \\
    &+\avr{\rho}\avf{S''_fg''} +\avr{\rho}\avf{S''_gf''} \\
    &-\avr{g''}\dv{\avr{\mathbf{F}}}-\avr{f''}\dv{\avr{\mathbf{G}}}\;,
  \end{split}
\end{equation}
having used the relationship
\begin{equation}
  \avr{\rho fg\mathbf{v}} = \avr{\rho} \avf{f}\avf{g}\avf{\mathbf{v}}
  + \avr{\rho}\avf{\mathbf{v}''f''g''}+ \avf{f''g''}\avf{\mathbf{v}}
  + \avf{f''\mathbf{v}''}\avf{g}+ \avf{g''\mathbf{v}''}\avf{f}\;,
\end{equation}
together with
\begin{equation}
  \begin{split}
    \avr{g\dv{\mathbf{F}}}-\avf{g}\dv{\avr{\mathbf{F}}}
    &= \avr{g''\dv{\mathbf{F}'}}+\avr{g''}\dv{\avr{\mathbf{F}}} \\
    &= \dvp{\avr{\mathbf{F}'g''}}-\avr{\mathbf{F}'\cdot\nabla g''}+\avr{g''}\dv{\avr{\mathbf{F}}}
  \end{split}
\end{equation}
and similarly for the other second-order term $f\dv{\mathbf{G}}$.

\subsection{Reynolds Stresses}

Consider the momentum equations:
\begin{align}
  &f = u_i\;, && F_k=p\,\delta_{ik}-\tau_{ik}\;, && S_f = b\, g_i- \epsilon_{imk} f_m u_k\;,\\
  &g = u_j\;, && G_k=p\,\delta_{jk}-\tau_{jk}\;, && S_g = b\, g_j- \epsilon_{jmk} f_m u_k\;.
\end{align}
Substituting into Eq.~(\ref{equ:covfg}), we find
\begin{equation}\label{equ:Rij}
  \pt{R_{ij}}{t}  = C_{ij}  +P_{ij} -E_{ij}
  +\avr{\rho}^{-1} ( \Pi_{ij} - \pt{T_{ijk}}{k} + M_{ij} ) +B_{ij} -F_{ij}
\end{equation}
where
\begin{align}
  \tag{Reynolds-stress component}
    R_{ij}  &=\phantom{-}\avf{u_i''u_j''}\\
  \tag{mean advection}
    C_{ij}  &=-\avf{u}_k\,\pt{R_{ij}}{k}\\
  \tag{mean-gradient (shear) production}
    P_{ij}  &=-R_{ik} \pt{\avf{u}_j}{k} -R_{jk} \pt{\avf{u}_i}{k} \\
  \tag{viscous dissipation}
    E_{ij}  &=\phantom{-} \avr{\rho}^{-1} ( \avr{\tau'_{jk} \pt{u''_i}{k}}
                                           +\avr{\tau'_{ik} \pt{u''_j}{k}} )\\
  \tag{turbulent transport}
    T_{ijk} &=\phantom{-} \avr{\rho u''_i u''_j u''_k}
                         +\avr{p'u'_i}\delta_{jk} +\avr{p'u'_j}\delta_{ik}
                         -( \avr{\tau'_{jk} u''_i} +\avr{\tau'_{ik} u''_j} )\\
  \tag{pressure strain}
    \Pi_{ij}&=\phantom{-} \avr{ p' ( \pt{u''_i}{j}+ \pt{u''_j}{i} ) }\\
  \tag{mean flux}
    M_{ij}  &=\phantom{-} \avr{u''_i} ( \pt{\bar{\tau}_{jk}}{k} - \pt{\bar{p}}{j} )
                         +\avr{u''_j} ( \pt{\bar{\tau}_{ik}}{k} - \pt{\bar{p}}{i} )\\
  \tag{buoyancy production-destruction}
    B_{ij}  &=\phantom{-} \avf{b''u''_j}\,g_i+\avf{b''u''_i}\,g_j \\
  \tag{Coriolis redistribution}
    F_{ij}  &=\phantom{-} \epsilon_{imk}f_m R_{jk} +\epsilon_{jmk}f_m R_{ik}
\end{align}
Depending on symmetries, several terms can be zero (within statistical convergence). Note that if $b\equiv 1$, then $B_{ij}=0$. The mean flux term is sometimes written as $M_{ij}=D_{ij}-G_{ij}$, the first term grouping the mean viscous stress contributions and the last term the mean pressure contributions. In cases of constant density, then $\avr{u''_j}=0$ and $M_{ij}=D_{ij}=G_{ij}=0$.

Contracting indices, the budget equation for the turbulent kinetic energy $K=R_{ii}/2$ reads
\begin{equation}
  \pt{K}{t} = C +P +B -E +\avr{\rho}^{-1} \left( \Pi -\pt{T_k}{k} +M \right)\;.
\end{equation}
Note that $F_{ii}=0$. If the flow is solenoidal, then $\Pi=0$.

\subsection{Scalar Fluxes}

Consider the momentum and scalar equations:
\begin{align}
  &f = u_i\;,   && F_k =p\,\delta_{ik}-\tau_{ik}\;,  && S_f = b\, g_i- \epsilon_{imk} f_m u_k\;,\\
  &g = \zeta\;, && G_k =-q_{k}\;,                    && S_g = w\;.
\end{align}
Substituting into Eq.~(\ref{equ:covfg}), we find
\begin{equation}\label{equ:Ris}
  \pt{R_{i \zeta}}{t}= C_{i\zeta} +P_{i\zeta} -E_{i\zeta}
  +\avr{\rho}^{-1} ( \Pi_{i\zeta} -\pt{T_{i\zeta k}}{k} +M_{i\zeta} ) +B_{i\zeta} -F_{i\zeta} +Q_{i\zeta}
\end{equation}
where
\begin{align}
  \tag{scalar-flux component}
    R_{i\zeta}    &=\phantom{-} \avf{u_i''\zeta''} \\
  \tag{mean advection}
    C_{i\zeta}    &=-\avf{u}_k\,\pt{R_{i\zeta}}{k} \\
  \tag{mean-gradient and tilting production}
    P_{i\zeta}    &=-R_{ik}\,\pt{\avf{\zeta}}{k} -R_{\zeta k}\,\pt{\avf{u}_i}{k} \\
  \tag{molecular destruction}
    E_{i\zeta} &=\phantom{-} \avr{\rho}^{-1} ( \avr{q_k'\pt{u_i''}{k}}
                                              +\avr{\tau_{ik}' \pt{\zeta''}{k}} ) \\
  \tag{pressure-flux interaction}
    \Pi_{i\zeta}  &=\phantom{-} \avr{p'\pt{\zeta''}{i}}\\
  \tag{\text{turbulent transport}}
    T_{i\zeta k}  &=\phantom{-} \avr{\rho \zeta''u''_ku''_i } +\avr{p's''}\delta_{ik}
                               -\avr{\tau'_{ik}s''}-\avr{q_k'u_i''}\\
  \tag{mean flux}
    M_{i\zeta}    &=\phantom{-} \avr{u''_i}\pt{\avr{q}_k}{k}
                               +\avr{\zeta''} ( \pt{\bar{\tau}_{ik}}{k} -\pt{\bar{p}}{i} ) \\
  \tag{buoyancy iteraction}
    B_{i\zeta}    &=\phantom{-} \avf{b''\zeta''}g_i \\
  \tag{Coriolis interation}
    F_{i\zeta}    &=\phantom{-} \epsilon_{imk} f_m R_{k\zeta} \\
  \tag{source}
    Q_{i\zeta}    &=\phantom{-} \avf{w''u_i''}
\end{align}
The mean flux term is sometimes written as $M_{i\zeta}=D_{i\zeta}-G_{i\zeta}$, the first term grouping the mean molecular flux contributions and the last term the mean pressure contributions.

\subsection{Scalar Variance}

Consider the scalar equation twice:
\begin{align}
  &f = \zeta\;, && F_k=-q_{k}\;, && S_f = w\;,\\
  &g = \zeta\;, && G_k=-q_{k}\;, && S_g = w\;.
\end{align}
Substituting into Eq.~(\ref{equ:covfg}), we find
\begin{equation}\label{equ:Rss}
  \pt{R_{\zeta \zeta}}{t}= C_{\zeta\zeta}+P_{\zeta\zeta} - E_{\zeta\zeta}
  +\avr{\rho}^{-1} (- \pt{T_{\zeta\zeta k}}{k} +M_{\zeta\zeta} ) + Q_{\zeta\zeta}
\end{equation}
where
\begin{align}
  \tag{scalar variance}
    R_{\zeta \zeta} &=\phantom{-} \avf{\zeta''\zeta''}\\
  \tag{mean advection}
    C_{\zeta \zeta} &=-\avf{u}_k\,\pt{R_{\zeta \zeta}}{k}\\
  \tag{mean-gradient production}
    P_{\zeta \zeta} &=-2 R_{k \zeta}\,\pt{\avf{\zeta}}{k}\\
  \tag{molecular destruction}
    E_{\zeta\zeta} &=\phantom{-} 2 \avr{\rho}^{-1}\avr{q_k' \pt{\zeta''}{k}}\\
  \tag{turbulent transport}
    T_{\zeta\zeta k}&=\phantom{-} \avr{\rho \zeta''^{2} u''_k } -2 \avr{q_k'\zeta''}\\
  \tag{mean flux}
    M_{\zeta\zeta}  &=\phantom{-} 2 \avr{\zeta^{''}} \pt{\avr{q}_k}{k}\\
  \tag{source}
    Q_{\zeta\zeta}  &=\phantom{-} 2 \avf{w''\zeta''}
\end{align}

\subsection{Energy Equation}

To be developed.

\chapter{Code}\label{sec:code}

The root directory contains the sources for the common libraries, and the directory {\tt tools} contains the sources for the specific binaries: the main code in {\tt tools/dns}, and the preprocessing and postprocessing tools.

Files {\tt README} and {\tt TODO} contain additional information. To compile, read {\tt INSTALL}.

Directory {\tt examples} contains a few examples to get acquainted with using the code.

%%%%%%%%%%%%%%%%%%%%%%%%%%%%%%%%%%%%%%%%%%%%%%%%%%%%%%%%%%%%%%%%%%%%%%%%%%%%%%%%
%%%%%%%%%%%%%%%%%%%%%%%%%%%%%%%%%%%%%%%%%%%%%%%%%%%%%%%%%%%%%%%%%%%%%%%%%%%%%%%%
\section{Executables}

{
\centering
\setlength{\tabcolsep}{0pt}
\footnotesize

%%%%%%%%%%%%%%%%%%%%%%%%%%%%%%%%%%%%%%%%%%%%%%%%%%%%%%%%%%%%%%%%%%%%%%%%%%%%%%%%
\rowcolors{1}{gray!25}{gray!10}
%
\begin{longtable}{p{0.15\textwidth} p{0.85\textwidth}}
%
\multicolumn{2}{>{\columncolor{lightgreen}}c}{\rule{0pt}{11pt}\normalsize\bf Simulation}\\
%
\tt dns.x & 
main program used to run a simulation. It will read its input from the file
dns.ini that the user must supply. An example file is located in {\tt
  examples}. All standard output is written to dns.log and dns.out. Errors are
reported to dns.err and warnings to dns.war. In order to run the simulation you
must provide with an initial flow and scalar fields. and a grid file.\newline
Sources in {\tt tools/dns}.\\
\end{longtable}

%%%%%%%%%%%%%%%%%%%%%%%%%%%%%%%%%%%%%%%%%%%%%%%%%%%%%%%%%%%%%%%%%%%%%%%%%%%%%%%%
\rowcolors{1}{gray!25}{gray!10}
%
\begin{longtable}{p{0.15\textwidth} p{0.85\textwidth}}
%
\multicolumn{2}{>{\columncolor{lightgreen}}c}{\rule{0pt}{11pt}\normalsize\bf Preprocessing}\\
%
\tt inigrid.x & 
generates the grid by reading the parameters of the dns.ini
file, section [IniGridOx], [IniGridOy] and [IniGridOz].\newline
Sources in {\tt tools/initialize/grid}.\\
\tt inirand.x & 
generates the scal.rand or flow.rand file that contains a
pseudo-random, isotropic field that will be used by the following program to
generate flow or scalar initial fields. The parameters are described in
dns.ini, section [Broadband].\newline Sources in {\tt tools/initialize/rand}\\
\tt iniscal.x &
generates the scal.ics file by reading the parameters of the dns.ini file,
section [IniFields].\newline Sources in {\tt tools/initialize/scal.}\\
\tt iniflow.x &
generates the flow.ics file by reading the parameters of the dns.ini file,
section [IniFields].\newline Sources in {\tt tools/initialize/flow}.\\
\end{longtable}

%%%%%%%%%%%%%%%%%%%%%%%%%%%%%%%%%%%%%%%%%%%%%%%%%%%%%%%%%%%%%%%%%%%%%%%%%%%%%%%%
\rowcolors{1}{gray!25}{gray!10}
%
\begin{longtable}{p{0.15\textwidth} p{0.85\textwidth}}
%
\multicolumn{2}{>{\columncolor{lightgreen}}c}{\rule{0pt}{11pt}\normalsize\bf Postprocessing}\\
%
\tt averages.x & 
calculates main average profiles and conditional averages (outer intermittency).\newline
Sources in {\tt tools/statistics/averages}.\\
\tt spectra.x & 
calculates 1D, 2D and 3D spectra and co-spectra of main variables. Correlations should be
included here.\newline
Sources in {\tt tools/statistics/spectra}.\\
\tt pdfs.x & 
calculates PDFs, joints PDFs and conditional PDFs.\newline
Sources in {\tt tools/statistics/pdfs}.\\
\tt visuals.x & 
calculates different fields and exports data for visualization (default is
ensight format). \newline
Sources in {\tt tools/plot/visuals}.\\
\end{longtable}

}

\pagebreak

%%%%%%%%%%%%%%%%%%%%%%%%%%%%%%%%%%%%%%%%%%%%%%%%%%%%%%%%%%%%%%%%%%%%%%%%%%%%%%%%
%%%%%%%%%%%%%%%%%%%%%%%%%%%%%%%%%%%%%%%%%%%%%%%%%%%%%%%%%%%%%%%%%%%%%%%%%%%%%%%%
\section{Input file dns.ini}

The following tables describe the different blocks appearing in the input file {\tt dns.ini}. The first column contains the tag. The second column contains the possible values, the first one being the default one and the word {\it value} indicating that a numerical value needs to be provided. The third column describes the field. This data is read in the file {\tt *\_READ\_GLOBAL} and in the files {\tt *\_READ\_LOCAL} of each of the tools; the variable corresponding to each field should be also read there.

Data is case insensitive.

{
\centering
\setlength{\tabcolsep}{0pt}
\footnotesize

%%%%%%%%%%%%%%%%%%%%%%%%%%%%%%%%%%%%%%%%%%%%%%%%%%%%%%%%%%%%%%%%%%%%%%%%%%%%%%%%
\rowcolors{1}{gray!25}{gray!10}
%
\begin{longtable}{p{0.15\textwidth} p{0.3\textwidth} p{0.55\textwidth}}
%
\multicolumn{3}{>{\columncolor{lightblue}}c}{\normalsize\bf [Version]}\\
%
\tt Major & {\it value} &
Major version number. An error is generated if different from the value set in
{\tt DNS\_READ\_GLOBAL}.\\
\tt Minor & {\it value} &
Minor version number. A warning is generated if different from the value set in
{\tt DNS\_READ\_GLOBAL}.\\
\end{longtable}

%%%%%%%%%%%%%%%%%%%%%%%%%%%%%%%%%%%%%%%%%%%%%%%%%%%%%%%%%%%%%%%%%%%%%%%%%%%%%%%%
\rowcolors{1}{gray!25}{gray!10}
%
\begin{longtable}{p{0.15\textwidth} p{0.3\textwidth} p{0.55\textwidth}}
%
\multicolumn{3}{>{\columncolor{lightblue}}c}{\normalsize\bf [Main]}\\
%
\tt Type & \tt temporal, spatial      & 
Temporally evolving or spatially evolving simulation.\\
\tt Flow & \tt shear, jet, isotropic & 
Geometry of the flow, mainly related to initial and boundary conditions.\\
\tt CalculateFlow & \tt yes, no & 
Execute code segments affecting the flow variables.\\
\tt CalculateScalar & \tt yes, no & 
Execute code segments affecting the scalar variables.\\
\tt Equations & \tt internal, total, incompressible &
Define system of equations to be solved.\\
\tt Mixture & \tt None, AirVapor, AirWater, AirWaterSupersaturation & 
Defines the mixture to be used for the thermodynamics.\\
\tt TermAdvection & \tt divergence, convective, skewsymmetric &
Formulation of the advection terms.\\
\tt TermViscous & \tt divergence, explicit&
Formulation of the viscous terms.\\
\tt TermDiffusion & \tt divergence, explicit &
Formulation of the diffusion terms.\\
\tt TermBodyForce & \tt None, Explicit, Homogeneous, Linear, Bilinear,
Quadratic, PiecewiseLinear, PiecewiseBilinear &
Formulation of the body force terms (see routine {\tt flow/flow\_buoyancy}).\\
\tt TermCoriolis & \tt none, explicit, normalized &
Formulation of the Coriolis terms.\\
\tt TermRadiation & \tt None, Bulk1dGlobal, Bulk1dLocal, Bulk1dMixed,
Bulk1dLocalMap, Bulk1dMixedMap &
Formulation of the radiation termr (see routine {\tt flow/flow\_radiation}).\\
\tt SpaceOrder & \tt CompactJacobian4, CompactJacobian6, CompactDirect6 &
Finite difference method used for the spatial derivatives.\\
\tt TimeOrder & \tt RungeKuttaExplicit3, RungeKuttaExplicit4, RungeKuttaDiffusion3 &
Runge-Kutta method used for the time advancement.\\
\tt TimeStep & {\it value} &
If positive, constant time step to be used in the time marching scheme.\\
\tt TimeCFL & {\it value} & Courant number for the advection part.\\
\end{longtable}

%%%%%%%%%%%%%%%%%%%%%%%%%%%%%%%%%%%%%%%%%%%%%%%%%%%%%%%%%%%%%%%%%%%%%%%%%%%%%%%%
\rowcolors{1}{gray!25}{gray!10}
%
\begin{longtable}{p{0.15\textwidth} p{0.3\textwidth} p{0.55\textwidth}}
%
\multicolumn{3}{>{\columncolor{lightblue}}c}{\normalsize\bf [Iteration]}\\
%
\tt Start & {\em value} & Initial iteration. The corresponding files {\tt flow.*}
and {\tt scal.*} will be read from disk.\\
\tt End & {\em value} & Final iteration at which the algortihm will be stopped.\\
\tt Restart & {\em value} & Iteration step specifying the frequency with which to
write the restart files to disk.\\
\tt Statistics & {\em value} & Iteration step specifying the frequency with which to
calculate statistics.\\ 
\tt IteraLog& {\em value} & Iteration step specifying the frequency with which to
write the log-file {\tt dns.out}.\\
\tt RunAvera& \tt no, yes & Save running averages to disk (spatially evolving simulations).\\
\tt RunLines& \tt no, yes & Save line information to disk (spatially evolving simulations).\\
\tt RunPlane& \tt no, yes & Save plane information to disk (spatially evolving simulations).\\
\tt StatSave& {\em value} &  Iteration step specifying the frequency with which to
accumulate statistics (spatially evolving simulations).\\
\tt StatStep& {\em value} & Iteration step specifying the frequency with which to
save data to disk (spatially evolving simulations).\\
\end{longtable}

%%%%%%%%%%%%%%%%%%%%%%%%%%%%%%%%%%%%%%%%%%%%%%%%%%%%%%%%%%%%%%%%%%%%%%%%%%%%%%%%
\rowcolors{1}{gray!25}{gray!10}
%
\begin{longtable}{p{0.15\textwidth} p{0.3\textwidth} p{0.55\textwidth}}
%
\multicolumn{3}{>{\columncolor{lightblue}}c}{\normalsize\bf [Parameters]}\\
%
\tt Reynolds & {\em value} & Reynolds number $Re$ in
section~\ref{sec:equations}.\\
\tt Prandtl  & {\em value} & Prandtl number $Pr$ in
section~\ref{sec:equations}.\\
\tt Froude   & {\em value} & Froude number $Fr$ in
section~\ref{sec:equations}.\\
\tt Rossby   & {\em value} & Rossby number $Ro$ in
section~\ref{sec:equations}.\\
\tt Mach     & {\em value} & Mach number $Ma$ in
section~\ref{sec:equations}.\\
\tt Gama    & {\em value} & Ratio of specific heats $\gamma$ in
section~\ref{sec:equations}.\\
\tt Schmidt & {\em value1, value2, ...} & List of Schmidt numbers $Sc_i$ in
section~\ref{sec:equations}. The number of values defines the number of scalars
in the problem. If a mixture is defined in the block [Main], then consistency is
checked.\\
\tt Damkohler & {\em value1, value2, ...} & List of Damkohler numbers $Da_i$ in
section~\ref{sec:equations}.\\
\end{longtable}

%%%%%%%%%%%%%%%%%%%%%%%%%%%%%%%%%%%%%%%%%%%%%%%%%%%%%%%%%%%%%%%%%%%%%%%%%%%%%%%%
\rowcolors{1}{gray!25}{gray!10}
%
\begin{longtable}{p{0.15\textwidth} p{0.3\textwidth} p{0.55\textwidth}}
%
\multicolumn{3}{>{\columncolor{lightblue}}c}{\normalsize\bf [Control]}\\
%
\tt FlowLimit & \tt yes, no & Monitor and eventually force the thermodynamic
fields to be  within a prescribed interval.\\  
\tt MinPressure & {\em value} & Lower bound for the pressure interval.\\
\tt MaxPressure & {\em value} & Upper bound for the pressure interval.\\
\tt MinDensity & {\em value} & Lower bound for the density interval.\\
\tt MaxDensity & {\em value} & Upper bound for the density interval.\\
ScalLimit  & \tt yes, no & Monitor and eventually force the scalar fields to be
within a prescribed interval.\\ 
\tt MinScalar & {\em value} & Lower bound for the scalar interval.\\
\tt MaxScalar & {\em value} & Upper bound for the scalar interval.\\
\end{longtable}

%%%%%%%%%%%%%%%%%%%%%%%%%%%%%%%%%%%%%%%%%%%%%%%%%%%%%%%%%%%%%%%%%%%%%%%%%%%%%%%%
\rowcolors{1}{gray!25}{gray!10}
%
\begin{longtable}{p{0.15\textwidth} p{0.3\textwidth} p{0.55\textwidth}}
%
\multicolumn{3}{>{\columncolor{lightblue}}c}{\normalsize\bf [Grid]}\\
%
\tt Imax & {\it value} & Number of points along the Ox direction (first array
index).\\ 
\tt Jmax & {\it value} & Number of points along the Oy direction (second array
index).\\ 
\tt Kmax & {\it value} & Number of points along the Oz direction (third array
index). If set equal to 1, then 2D simulation.\\
\tt Imax(*) & {\it value} & Number of points per processor (MPI task) along the Ox
direction (MPI parallel mode).\\
\tt Jmax(*) & {\it value} & Number of points per processor (MPI task) along the Oy
direction (MPI parallel mode). So far, this value is set equal to the total size
because only a 2D decomposition has been implemented.\\
\tt Kmax(*) & {\it value} & Number of points per processor (MPI task) along the
Oz direction (MPI parallel mode).\\
\tt XUniform & \tt yes, no & Uniform grid is used in the Ox direction; no Jacobian
information is needed.\\
\tt YUniform & \tt yes, no & Uniform grid is used in the Oy direction; no Jacobian
information is needed.\\
\tt ZUniform & \tt yes, no & Uniform grid is used in the Oz direction; no Jacobian
information is needed.\\
\tt XPeriodic & \tt no, yes & Periodicity along Ox direction.\\
\tt YPeriodic & \tt no, yes & Periodicity along Oy direction.\\
\tt ZPeriodic & \tt no, yes & Periodicity along Oz direction.\\
\end{longtable}

%%%%%%%%%%%%%%%%%%%%%%%%%%%%%%%%%%%%%%%%%%%%%%%%%%%%%%%%%%%%%%%%%%%%%%%%%%%%%%%%
\rowcolors{1}{gray!25}{gray!10}
%
\begin{longtable}{p{0.15\textwidth} p{0.3\textwidth} p{0.55\textwidth}}
%
\multicolumn{3}{>{\columncolor{lightblue}}c}{\normalsize\bf [BoundaryConditions]}\\
%
ToBeFilled & & \\
%
\tt VelocityImin & \tt none, noslip, freeslip & Velocity boundary condition at $x_\text{min}$
(incompressible mode).\\
\tt VelocityImax & \tt none, noslip, freeslip & Velocity boundary condition at $x_\text{max}$
(incompressible mode).\\
\tt Scalar\#Imin & \tt none, dirichlet, neumman & Scalar boundary condition at $x_\text{min}$
(incompressible mode). The symbol {\tt \#} is the number of the scalar.\\
\tt Scalar\#Imax & \tt none, dirichlet, neumman & Scalar boundary condition at $x_\text{max}$
(incompressible mode). The symbol {\tt \#} is the number of the scalar.\\
\multicolumn{3}{>{\columncolor{gray!25}}l}{Similarly in the other two directions.}\\
\end{longtable}

%%%%%%%%%%%%%%%%%%%%%%%%%%%%%%%%%%%%%%%%%%%%%%%%%%%%%%%%%%%%%%%%%%%%%%%%%%%%%%%%
\rowcolors{1}{gray!25}{gray!10}
%
\begin{longtable}{p{0.15\textwidth} p{0.3\textwidth} p{0.55\textwidth}}
%
\multicolumn{3}{>{\columncolor{lightblue}}c}{\normalsize\bf [BufferZone]}\\
%
\tt Type & \tt none, relaxation, filter, both & Type of buffer or sponge layer
to use.\\
\tt LoadBuffer & \tt no, yes & If {\tt no}, then create reference buffer fields
from the current fields and save them to disk.\newline 
If {\tt yes}, then read the necessary buffer fields from disk. E.g., for the
upper boundary, the file name to be searched for would be {\tt flow.bcs.jmax}
and {\tt scal.bcs.jmax}.\\ 
\tt PointsImin & {\em value} & Number of points in the $Ox$ direction at $x_\text{min}$.\\
\tt PointsImax & {\em value} & Number of points in the $Ox$ direction at $x_\text{max}$.\\
\tt PointsUJmin & {\em value} & Number of points in the $Oy$ direction at
$y_\text{min}$ for the velocity fields.\\
\tt PointsUJmax & {\em value} & Number of points in the $Oy$ direction at
$y_\text{max}$ for the velocity fields.\\
\tt PointsEJmin & {\em value} & Number of points in the $Oy$ direction at
$y_\text{min}$ for the thermodynamic fields.\\
\tt PointsEJmax & {\em value} & Number of points in the $Oy$ direction at
$y_\text{max}$ for the thermodynamic fields.\\
\tt PointsSJmin & {\em value} & Number of points in the $Oy$ direction at
$y_\text{min}$ for the scalar fields.\\
\tt PointsSJmax & {\em value} & Number of points in the $Oy$ direction at
$y_\text{max}$ for the scalar fields.\\
\tt ParametersU & {\em value1, value2, ...} & Set of parameters defining the
the strength and the exponent of the relaxation term in the flow and
thermodynamic fields, section~\ref{sec:buffer}.\\
\tt ParametersS & {\em value1, value2, ...} & Set of parameters defining the
the strength and the exponent of the relaxation term in the scalar fields,
section~\ref{sec:buffer}.\\
\end{longtable}

%%%%%%%%%%%%%%%%%%%%%%%%%%%%%%%%%%%%%%%%%%%%%%%%%%%%%%%%%%%%%%%%%%%%%%%%%%%%%%%%
\rowcolors{1}{gray!25}{gray!10}
%
\begin{longtable}{p{0.15\textwidth} p{0.3\textwidth} p{0.55\textwidth}}
%
\multicolumn{3}{>{\columncolor{lightblue}}c}{\normalsize\bf [Flow]}\\
%
\tt Pressure  & {\em value} & Reference mean pressure.\\
\tt Density   & {\em value} & Reference mean density.\\  
\tt VelocityX & {\em value} & Reference mean velocity along $Ox$.\\
\tt VelocityY & {\em value} & Reference mean velocity along $Oy$.\\
\tt VelocityZ & {\em value} & Reference mean velocity along $Oz$.\\
%
\tt ProfileVelocity & \tt None, Linear, Tanh, Erf, Ekman, EkmanP &
Function form of the mean velocity profile, typically along the direction
$Ox$.\\
\tt YCoorVelocity & {\em value} & Coordinate along $Oy$ of the reference point
of the profile, relative to the total scale, equation~(\ref{equ:profile}).\\
\tt ThickVelocity & {\em value} & Reference profile thickness, equation~(\ref{equ:profile}).\\
\tt DeltaVelocity & {\em value} & Reference profile difference,
equation~(\ref{equ:profile}).\\
\tt DiamVelocity  & {\em value} & Reference profile diameter (jet mode).\\ 
%
\multicolumn{3}{l}{Similarly for density or temperature}\\
%
\end{longtable}

%%%%%%%%%%%%%%%%%%%%%%%%%%%%%%%%%%%%%%%%%%%%%%%%%%%%%%%%%%%%%%%%%%%%%%%%%%%%%%%%
\rowcolors{1}{gray!25}{gray!10}
%
\begin{longtable}{p{0.15\textwidth} p{0.3\textwidth} p{0.55\textwidth}}
%
%
\multicolumn{3}{>{\columncolor{lightblue}}c}{\normalsize\bf [Scalar]}\\
%
\tt ProfileScalar\# & \tt None, Linear, Tanh, Erf, LinearErf &
Function form of the mean profile.\\
\tt MeanScalar\# & {\em value} & Reference mean scalar.\\
\tt YCoorScalar\# & {\em value} & Coordinate along $Oy$ of the reference point
of the profile, relative to the total scale.\\
\tt ThickScalar\# & {\em value} & Reference profile thickness.\\
\tt DeltaScalar\# & {\em value} & Reference profile difference.\\
\tt DiamScalar\#  & {\em value} & Reference profile diameter (jet mode).\\ 
%
\end{longtable}

%%%%%%%%%%%%%%%%%%%%%%%%%%%%%%%%%%%%%%%%%%%%%%%%%%%%%%%%%%%%%%%%%%%%%%%%%%%%%%%%
\rowcolors{1}{gray!25}{gray!10}
%
\begin{longtable}{p{0.15\textwidth} p{0.3\textwidth} p{0.55\textwidth}}
%
\multicolumn{3}{>{\columncolor{lightblue}}c}{\normalsize\bf [BodyForce]}\\
%
\tt Vector     & {\em value1, value2, value3} & Components of the buoyancy
unitary vector $(g_1,\,g_2,\,g_3)$ in section~\ref{sec:equations}.\\
\tt Parameters & {\em value1, value2, ...} & Set of parameters defining the
buoyancy function $b^e(s_i)$.\\
\end{longtable}

%%%%%%%%%%%%%%%%%%%%%%%%%%%%%%%%%%%%%%%%%%%%%%%%%%%%%%%%%%%%%%%%%%%%%%%%%%%%%%%%
\rowcolors{1}{gray!25}{gray!10}
%
\begin{longtable}{p{0.15\textwidth} p{0.3\textwidth} p{0.55\textwidth}}
%
\multicolumn{3}{>{\columncolor{lightblue}}c}{\normalsize\bf [Rotation]}\\
%
\tt Vector     & {\em value1, value2, value3} & Components of the angular
velocity vector $(f_1,\,f_2,\,f_3)$ in section~\ref{sec:equations}.\\
\tt Parameters & {\em value1, value2, ...} & Set of parameters the Coriolis
force term.\\
\end{longtable}

%%%%%%%%%%%%%%%%%%%%%%%%%%%%%%%%%%%%%%%%%%%%%%%%%%%%%%%%%%%%%%%%%%%%%%%%%%%%%%%%
\rowcolors{1}{gray!25}{gray!10}
%
\begin{longtable}{p{0.15\textwidth} p{0.3\textwidth} p{0.55\textwidth}}
%
\multicolumn{3}{>{\columncolor{lightblue}}c}{\normalsize\bf [Radiation]}\\
%
\tt Scalar & {\em value} & Index of scalar field on which the effect of 
radiation heating or cooling is acting.\\
\tt Parameters & {\em value1, value2, ...} & Set of parameters defining the
radiation function $r^e(s_i)$.\\
\end{longtable}

%%%%%%%%%%%%%%%%%%%%%%%%%%%%%%%%%%%%%%%%%%%%%%%%%%%%%%%%%%%%%%%%%%%%%%%%%%%%%%%%
\rowcolors{1}{gray!25}{gray!10}
%
\begin{longtable}{p{0.15\textwidth} p{0.3\textwidth} p{0.55\textwidth}}
%
\multicolumn{3}{>{\columncolor{lightblue}}c}{\normalsize\bf [IniFields]}\\
%
\tt Velocity & \tt None, VelocityDiscrete, VelocityBroadband,
VorticityBroadband, PotentialBroadband & Type of initial velocity field.\\
\tt Temperature & \tt None, PlaneBroadband, PlaneDiscrete &
Type of initial temperature field.\\
\tt Scalar & \tt None, LayerDiscrete, LayerBroadband, PlaneDiscrete,
PlaneBroadband, DeltaDiscrete, DeltaBroadband, FluxDiscrete, FluxBroadband &
Type of initial scalar field.\\
\tt ForceDilatation & \tt yes, no & Force the velocity field to satisfy the
solenoidal constraint.\\
\tt ThickIni & {\em value}[{\em1 , value2, ...}]& Thickness of fluctuation shape
profile. The mean profile is set by the corresponding values in {\tt [Flow]} and
{\tt [Scalar]}. In case of the scalar, as many values as scalars should be
provided. You can differentiate between flow and scalar values by using {\tt ThickIniK} and {\tt ThickIniS}.\\
\tt YCoorIni & {\em value}[{\em1 , value2, ...}] & Coordinate along $Oy$ of the
reference point of the fluctuation shape profile, relative to the total
scale. The mean profile is set by the corresponding values in {\tt [Flow]} and
{\tt [Scalar]}. In case of the scalar, as many values as scalars should be
provided. You can differentiate between flow and scalar values by using {\tt YCoorIniK} and {\tt YCoorIniS}. The default values are those specified in {\tt [Flow]} and
{\tt [Scalar]}.\\
\tt NormalizeK & {\em value} & Maximum value of the profile of the turbulent
kinetic energy.\\
\tt NormalizeP & {\em value} & Maximum value of the profile of the pressure
root-mean-square.\\
\tt NormalizeS & {\em value1, value2, ...} & Maximum value of the profile of the
scalar
root-mean-square.\\
\tt Mixture & \tt None, Equilibrium, LoadFields & Type of mixture with which to
initialize the thermodynamic fields.\\
\end{longtable}

%%%%%%%%%%%%%%%%%%%%%%%%%%%%%%%%%%%%%%%%%%%%%%%%%%%%%%%%%%%%%%%%%%%%%%%%%%%%%%%%
\rowcolors{1}{gray!25}{gray!10}
%
\begin{longtable}{p{0.15\textwidth} p{0.3\textwidth} p{0.55\textwidth}}
%
\multicolumn{3}{>{\columncolor{lightblue}}c}{\normalsize\bf [Broadband]}\\
%
\tt Type & \tt None, Physical, Phase & The random magnitude is set in physical
space or phase in frequency space. \\
\tt Distribution & uniform, gaussian & Type of the PDF.\\
\tt Seed & {\it value} & Seed for the random generator.\\
\tt Covariance & {\it value1, value2, ...} & Flow covariance matrix.\\ 
\tt Spectrum & \tt uniform, quadratic, quartic, gaussian & Form of the power
spectral density, equation~(\ref{equ:spectra}).\\
\tt f0 & {\it value} & Parameters defining the functional form of the power
spectral density.\\
\end{longtable}

%%%%%%%%%%%%%%%%%%%%%%%%%%%%%%%%%%%%%%%%%%%%%%%%%%%%%%%%%%%%%%%%%%%%%%%%%%%%%%%%
\rowcolors{1}{gray!25}{gray!10}
%
\begin{longtable}{p{0.15\textwidth} p{0.3\textwidth} p{0.55\textwidth}}
%
\multicolumn{3}{>{\columncolor{lightblue}}c}{\normalsize\bf [Discrete]}\\
%
\tt Type        &  \tt Varicose, Sinuous, Gaussian & Form of the perturbation.\\
\tt 2DAmpl      & {\it value1, value2, ...} & Amplitude of 2D modes. The number of
values sets the number of modes, beginning from the first.\\
\tt 2DPhi       & {\it value1, value2, ...} & Corresponding phases.\\
\tt Broadening  & {\it value} & Lateral extension of the perturbation.\\
\tt XLength     & {\it value} & In spatial simulations, longitudinal extension
of the inflow perturbation.\\
\end{longtable}

%% \item \textbf{3DAmpl} (\textit{A3d}): intensity. 
%% \item \textbf{3DXPhi} (\textit{Phix3d}): Same as above, but for the 3D pertubations.
%% \item \textbf{3DZPhi} (\textit{Phiz3d}): Same as above.

%%%%%%%%%%%%%%%%%%%%%%%%%%%%%%%%%%%%%%%%%%%%%%%%%%%%%%%%%%%%%%%%%%%%%%%%%%%%%%%%
\rowcolors{1}{gray!25}{gray!10}
%
\begin{longtable}{p{0.15\textwidth} p{0.3\textwidth} p{0.55\textwidth}}
%
\multicolumn{3}{>{\columncolor{lightblue}}c}{\normalsize\bf [PostProcessing]}\\
%
\tt Files     & {\it value1, value2, ...} & Iterations to be postprocessed.\\
\tt Subdomain & $i_{1}, i_{2}, j_{1}, j_{2}, k_{1}, k_{2}$ & 
Grid block to be postprocessed.\\
\tt Partition & $\alpha_1$[, $\alpha_2$], $\beta_1$, ..., $\beta_{n-1}$& 
Type of partition defined by values \{$\alpha_1$[, $\alpha_2$]\}. The first
parameter defines the conditioning field: 1. external field, 2. scalar field,
3. enstrophy, 4. magnitude of scalar gradient. The second parameter chooses
between a relative or an absolute threshold values. Set of thresholds
\{$\beta_1$, ...,$\beta_{n-1}$\} to define the partition of the conditioning
field into $n$ zones.\\ 
\tt ParamAverages & $\alpha_1$[, $\alpha_2$, $\alpha_3$, $\alpha_4$]& Main option
$\alpha_1$ (see {\tt tools/statistics/averages/averages.f90}); block size $\alpha_2$; gate
level $\alpha_3$; maximum order of the moments $\alpha_4$.\\
\tt ParamPdfs & $\alpha_1$[, $\alpha_2$, $\alpha_3$, $\alpha_4$]& Main option
$\alpha_1$ (see {\tt tools/statistics/pdfs/pdfs.f90}); block size $\alpha_2$; gate
level $\alpha_3$; number of bins $\alpha_4$.\\
\tt ParamSpectra & $\alpha_1$[, $\alpha_2$, $\alpha_3$, $\alpha_4$]& Main option
$\alpha_1$ (see {\tt tools/statistics/spectra/spectra.f90}); block size
$\alpha_2$; save full spectra $\alpha_3$; average over iteration $\alpha_4$.\\
\end{longtable}

% %%%%%%%%%%%%%%%%%%%%%%%%%%%%%%%%%%%%%%%%%%%%%%%%%%%%%%%%%%%%%%%%%%%%%%%%%%%%%%%%
\rowcolors{1}{gray!25}{gray!10}
%
\begin{longtable}{p{0.15\textwidth} p{0.3\textwidth} p{0.55\textwidth}}
%
\multicolumn{3}{>{\columncolor{lightblue}}c}{\normalsize\bf [Inflow]}\\
%
\tt Type & \tt None, Discrete, BroadbandPeriodic,  BroadbandSequential & Type of
inflow forcing to use 
in a spatially evolving simulation.\\
\tt Adapt & \it value & Interval in global time units for starting the inflow
forcing.\\ 
\end{longtable}

}

% {\Large\bf [Statistics]}

% \begin{itemize}
% \item \textbf{FilterEnergy} (\textit{ffltdmp}):  If ``yes'' calculate corresponding statistics.(default=no)
% \item \textbf{EpsInter} (\textit{eps\_inter}): (default=0.01)
% \item \textbf{IAvera} (\textit{statavg}): Planes i constant to be averaged.(default=1)
% \item \textbf{ILines} (\textit{statlin\_i}): I position of line to be saved. It has to match with JLines.(default=1)
% \item \textbf{JLines} (\textit{statlin\_j}): J position of line to be saved. It has to match with ILines. (default=1)
% \item \textbf{IPlane} (\textit{statpln}): Planes i constant to be saved. (default=1)
% \end{itemize}

% %%%%%%%%%%%%%%%%%%%%%%%%%%%%%%%%%%%%%%%%%%%%%%%%%%%%%%%%%%%%%%%%%%%%%%%%%%%%%%%%
% {\Large\bf [LES]}

% \begin{itemize}
% \item \textbf{Active} (\textit{iles}): ``yes'' if SGS terms are used. (default=no)
% \item \textbf{Transport}  (\textit{iles\_type\_tran}): ``none'', ``ssm'' for scale similarity , ``arm-rans'' for approximate reconstruction, ``arm-sptl'' for approximate reconstruction, ``arm-invs'' (default=ssm)
% \item \textbf{Regularization} (\textit{iles\_type\_regu}): ``none'', ``smg-static'', ``smg-dynamic'', ``smg-static-rms'', ``smg-dynamic-rms'' (default=smg-static)
% \item \textbf{Dissipation} (\textit{iles\_type\_diss}): ``none'', ``diss-sgs'' (default=none)
% \item \textbf{Chemistry} (\textit{iles\_type\_chem}):  ``none'', ``arm\_bs arm\_ps'' (default=none)
% \item \textbf{Filter0Size} (\textit{isgs\_f0size}): $\Delta\_f / \Delta\_g$ of filter level 0. Integer even number. (default=2)
% \item \textbf{Filter1Size} (\textit{isgs\_f1size}): $\Delta\_f / \Delta\_g$ of filter level 1. Integer even number. (default=4)
% \item \textbf{Inviscid} (\textit{iles\_inviscid}): (default=no)
% \item \textbf{JmaxDeact} (\textit{iles\_jmaxdeact}): (default=1)
% \item \textbf{SmagVariant} (\textit{sgs\_devsmag}): ``deviatoric'' for Smagorinsky expression in terms of the deviatoric tensors (``other'' for relation in the whole tensor). (default=deviatoric)
% \item \textbf{SmagTrans} (\textit{sgs\_smagtrans}): (default=-1.0)
% \item \textbf{SmagDelta} (\textit{sgs\_smagdelta}): (default=0.0)
% \item \textbf{Alpha} (\textit{sgs\_alpha}): Constant for scale similarity model.(default=1.0)
% \item \textbf{Csm} (\textit{sgs\_csm}): Smagorinsky constant.(default=0.13)
% \item \textbf{Prs} (\textit{sgs\_prs}): Smagorinsky Prandtl number. (default=0.6)
% \item \textbf{Sct} (\textit{sgs\_sct}): Smagorinsky Schmidt number. (default=0.6)
% \item \textbf{AlphaDil} (\textit{sgs\_pdil}): (default=2.2)
% \item \textbf{ChemistryCondDissipation} (\textit{iles\_type\_disZchem}): Conditonal dissipation model. ``mean'', ``one-dimensional'' (default=mean)
% \item \textbf{ChemistryDissipation} (\textit{iles\_type\_dischem}): Dissipation model. ``gradient'', ``diss-sgs'' (default=gradient)
% \item \textbf{FDFfile} (\textit{les\_fdf\_bs\_file}): file name
% \item \textbf{ChemistryVariance} (\textit{iles\_type\_recchem}): ssm arm-invs arm-rans arm-sptl (default=arm-rans)
% \item \textbf{ARM\_Spectrum} (\textit{iarm\_spc}): Type of model spectrum; ``Pope'' or ``adhoc''. (default=Pope)
% \item \textbf{ARM\_NL} (\textit{iarm\_nl}): Number of points in gamma for the c0-table.(default=10)
% \item \textbf{ARM\_NRe} (\textit{iarm\_nre}): Number of points in Reynolds for the c0-table. (default=10)
% \item \textbf{ARM\_ReMin} (\textit{arm\_remin}): Minimun Reynolds for the c0-table (default=10.0)
% \item \textbf{ARM\_DeltaRe} (\textit{arm\_dre}): Increment in the Reynolds for the c0-table.(default=10.0)
% \item \textbf{ARM\_ActivaT} (\textit{arm\_tact}): (default=100)
% \item \textbf{ARM\_FlameT} (\textit{arm\_tflame}): (default=10)
% \item \textbf{ARM\_StechioZ} (\textit{arm\_zst}): (default=0.2)
% \item \textbf{ARM\_SmoothZ} (\textit{arm\_smooth}): (default=0.1)
% \item \textbf{ARM\_c0} (\textit{arm\_c0\_inviscid}): (default=4.72)
% \end{itemize}

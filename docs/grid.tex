\chapter{Grid}\label{sec:grid}

The equations are solved using Cartesian co-ordinates and the grid is structured. The grid is constructed by building up the three directions separately (in a 2D case, $Oz$ has simply one node).  Each direction is broken into segments, and each of those segments is built with specified generation algorithms. The first node in each direction is set at zero. 

The executable is {\tt inigrid.x} and the data in {\tt dns.ini} is specified in the blocks [IniGridOx], [IniGridOy] and [IniGridOz]. Once created, basic information about the grid is saved into the file {\tt grid.sts}. Grid transformations can be done with {\tt transgrid.x}; the corresponding sources are in the {\tt tools/transform} sub-directory. 

%They allow to print out an ASCII file with the grid positions, add an offset, drop or introduce planes and make a scaling.  (As in the case of the utilities of the main code, a browse through them is recommended, though they are really simple.)

{
\centering
\setlength{\tabcolsep}{0pt}
\footnotesize

\rowcolors{1}{gray!25}{gray!10}
%
\begin{longtable}{p{0.15\textwidth} p{0.3\textwidth} p{0.55\textwidth}}
%
\multicolumn{3}{>{\columncolor{lightblue}}c}{\normalsize\bf [IniGridOx]}\\
%
\tt Segments & {\it value} & Number of segments in that direction.\\
\tt Periodic & \tt yes, no & If yes, a uniform mesh is done with a number of points equal to
  the sum of the points of all segments, and over a length equal to the sum of
  the length of all segments, regardless the input options for each segment. The
  last plane in that direction is dropped, so that the last point does not match
  the scale (the latter is a bit larger).\\
\tt Mirrored & \tt yes, no &  If yes, this direction is created with the corresponding options
  and then it is mirrored with respect to the origin. The final scale is the
  double of that set in the input file {\tt dns.ini} and the first node is
  moved to zero. In the case of mirroring the number of points is always even.\\
\tt Scales\_\# & {\it value} & Physical end of the segment number {\tt \#}.\\
\tt Points\_\# & {\it value} & Number of points in the segment number {\tt \#}. This number includes the first and the last points of the present segment, which are common with the adjacent  ones. (E.g., if one direction has three segments with 11, 16 and 6 points each, the total number of points in that direction will be 31, 30 steps.)\\
\tt Opts\_\#  &  {\it value1, value2, ...} & List of options for the generation algorithm:
\begin{enumerate}[leftmargin=*,noitemsep,topsep=0pt,parsep=0pt,partopsep=0pt]
\item[0] Uniform grid.
\item[4] Geometric progression. Not used in a loooong time.
\item[5] Explicit mapping: hyperbolic tangent of the space step.
\item[6] Explicit mapping: hyperbolic tangent of the stretching factor.
\end{enumerate}\\
\tt Vals\_\#  &  {\it value1, value2, ...} & List of constants for the different generation algorithms.\\
\end{longtable}

}

\section{Generation algorithms}

\subsection{Explicit mappings}
A basic grid is considered first with uniform spacing, that is
$\{s_j=h_0(j-1):\, j = 1,\ldots,n\}$. 

In case of a hyperbolic tangent (i.e. {\tt opts=5}), the mapping is
\begin{equation}
\frac{dx}{ds} = 1 + \frac{h_1/h_0-1}{2}\left[ 1 + \tanh\left(\frac{s-s_{t1}}{2\delta_1}\right)\right]
\end{equation}
that is, the grid step $\Delta x =dx/dj$ varies between the uniform values $h_0$
and $h_1$, the transition occurring at $s=s_{t1}$ and over a length $\delta_1$. The
space step $h_0$ corresponds to that of the initial uniform grid. The parameters
$s_{t1}$, $h_1/h_0$ and $\delta_1$ are provided in the corresponding {\tt vals}
record of the {\tt dns.ini} file.

The mapping can be written explicitly as
\begin{equation}
  x = s + (h_1/h_0-1)\delta_1\left\{\ln(1+\exp[(s-s_{t1})/2])+C\right\}
\end{equation}
where $C$ is the integration constant, defined such that $x=0$ for $s=0$. The final extension obeys the relationship
\begin{equation}
  x_\mathrm{max}=s_\mathrm{max} + (h_1/h_0-1)\delta_1\ln\left[\frac{\exp(s_\mathrm{max}/\delta_1)+1}{1+\exp(-s_{t1}/\delta_1)}\right],
\end{equation}
where $s_\mathrm{max} =(n-1)h_0$.

Two transitions are possible if the {\tt opts} record is set to $5,2$. The
mapping is then
\begin{equation}
\frac{dx}{ds} = 1 
+ \frac{h_1/h_0-1}{2}\left[ 1 + \tanh\left(\frac{s-s_{t1}}{2\delta_1}\right)\right]
+ \frac{h_2/h_0-1}{2}\left[ 1 + \tanh\left(\frac{s-s_{t2}}{2\delta_2}\right)\right]\;.
\end{equation}

In case of {\tt opts=6} a hyperbolic tangent variation of the stretching factor
$f=d/dx (\Delta x)$ (so defined, $f+1$ an approximation to the ratio $(\Delta x)
_{j+1}/(\Delta x) _{j} $) is imposed by solving the linear equation
\begin{equation}
\frac{d^2x}{ds^2} - (f/h_0)\frac{dx}{ds} = 0 \;,
\end{equation}
where $f$ is given by
\begin{equation}
f = \frac{\Delta f_1}{2}\left[ 1 + \tanh\left(\frac{s-s_{t1}}{2\delta_1}\right)\right]
\end{equation}
and the parameters $s_{t1}$, $\Delta f_1$ and $\delta_1$ are provided in the list
${\tt vals}$. The non-dimensional parameter $f$ then varies from $0$ to $\Delta
f_1$; values smaller that $0.1$ are recommended (less that 10\% stretching). A
first integral of this problem leads to
\begin{equation}
\frac{dx}{ds} =
\left[1+\exp\left(\frac{s-s_{t1}}{\delta_1}\right)\right]^{\delta_1(\Delta f_1/h_0)} \;.
\end{equation}
This equation for $x(s)$ needs to be solved numerically, but already shows that
this mapping leads to an exponential growth of the space step $\Delta x$ and the
grid $x(s)$. Note that $\delta_1$ admits negative values.

As before, two transitions are possible if the {\tt opts} record is set to $6,2$,
\begin{equation}
  f = \frac{\Delta f_1}{2}\left[ 1 + \tanh\left(\frac{s-s_{t1}}{2\delta_1}\right)\right]
  + \frac{\Delta f_2}{2}\left[ 1 + \tanh\left(\frac{s-s_{t2}}{2\delta_2}\right)\right] \;,
\end{equation}
which implies
\begin{equation}
  \frac{dx}{ds} =
  \left[1+\exp\left(\frac{s-s_{t1}}{\delta_1}\right)\right]^{\delta_1(\Delta f_1/h_0)}
  \left[1+\exp\left(\frac{s-s_{t2}}{\delta_2}\right)\right]^{\delta_2(\Delta f_2/h_0)}\;.
\end{equation}

\subsection{Geometric progression}
For each segment $i$ the geometric ratio $r_i$ is given (input variable
$val1$). The rest of the constrains are the number of points in each segment,
$n_i$ and the total length in that direction, $L$.  The first step of the first
segment is initialized to 1 (this number does not matter because of the final
scaling), and then the first segment is generated. The last step size is taken
as the first one for the second segment, and the sequence continue until the
last segment. A final rescaling adjusts the physical length of the grid in that
direction.

The length of each segment in the grid is saved into {\tt grid.sts}. For
calculating them, if we denote the length of each segment by $l_i$, we have
\begin{equation}
l_i = h_i^1(1+r_i+r_i^2+\ldots+r_i^{n_i-2})=
h_i^1\frac{1-r_i^{n_i-1}}{1-r_i}=h_i^1C_i
\end{equation}
The first step $h_i^1$ of each segment is related to the previous one by
\begin{equation}
h_{i+1}^1 = h_i^1 r_i^{n_i-1}
\end{equation}
and the equation to close the problem is
\begin{equation}
L=\sum_{seg}l_i=h_1^1\left( C_1 + r_1^{n_1-1}C_2 + 
r_1^{n_1-1}r_2^{n_2-1}C_3 + \ldots \right)
\end{equation}

